\begin{savequote}[75mm] 
Nulla facilisi. In vel sem. Morbi id urna in diam dignissim feugiat. Proin molestie tortor eu velit. Aliquam erat volutpat. Nullam ultrices, diam tempus vulputate egestas, eros pede varius leo.
\qauthor{Quoteauthor Lastname} 
\end{savequote}

\chapter{Introduction}

Une visite touristique est habituellement magnifiée avec un guide. Ce guide peut être un humain ou bien un outil électronique, comme les audio/visio guides. De nouvelles pratiques peuvent émerger des avancées sur ces guides numériques, en particulier avec des détecteurs de mouvements, les caméras et les écrans. Ces éléments peuvent permettre d'enrichir les visites en proposant des informations diverses sur des œuvres rencontrées.

Parmi les nouvelles pratiques~\cite{andr2014}, nous pouvons imaginer la reconnaissance des œuvres pour des visites sensibles au contexte, en utilisant par exemple de la réalité augmentée ou d'autres méthodes d'enrichissement de visite. De tels systèmes, dépendants du contexte du visiteur lors de sa visite, reposent sur des systèmes capables de reconnaître l'environnement. Pour les musées, les besoins de reconnaissance des œuvres ou de localisation, utilisant des modèles a priori de cet environnement, émergent. De tels modèles sont par exemple un ensemble de données annotées, images ou vidéos.

Notre objectif est ici de fournir des collections de données muséales, composées d'images fixes et de vidéos de visites, qui permettent de tester des approches de recherche d'information {\it ad-hoc} dans le cas d'images photographiques et de vidéos. Afin de couvrir des données différentes, chacune de ces collections porte sur un type de musée spécifique : l'une est un musée d'art, et la seconde porte sur l'archéologie. Une telle diversité est nécessaire pour étudier la robustesse des approches de recherche d'images et de vidéos. Afin d'éviter toute confusion dans la suite, nous appelons :
\begin{itemize}
\item {\bf collection} l'ensemble des données fournies (requêtes, corpus, vérité terrain, métadonnées éventuelles), et 
\item {\bf corpus} l'ensemble des images ou vidéos, avec les métadonnées permettant d'identifier de manière unique les œuvres qui y sont visibles. C'est sur ce corpus que les requêtes sont évaluées.
\end{itemize}

Le travail décrit ici est réalisé dans le contexte de plusieurs projets industriels dans lesquels de tels corpus sont utilisés. Ces projets visent à définir de nouveaux modes d'accès à l'information muséale. Pour cela, ils reposent fortement sur les outils d'indexation et de recherche d'images. Nous proposons donc de mettre à la disposition de la communauté les corpus proposés, dans le cadre de l'évaluation de la recherche de données visuelles.




\section{Objet, classes et instances}



\section{Corpus d'image muséales}


Dans cet article, quand nous parlons de musée, nous utilisons la définition du conseil international des musées\footnote{http://icom.museum/la-vision/definition-du-musee/L/2/} qui fait référence :
``Un musée est une institution permanente sans but lucratif au service de la société et de son développement ouverte au public, qui acquiert, conserve, étudie, expose et transmet le patrimoine matériel et immatériel de l’humanité et de son environnement à des fins d'études, d'éducation et de délectation''. Un musée peut présenter des objets très divers, comme la collection Burrell de Glasgow\footnote{http://www.glasgowlife.org.uk/museums/burrell-collection/Pages/default.aspx} ou bien plus focalisée, comme le musée van Gogh à Amsterdam\footnote{https://www.vangoghmuseum.nl}. Afin de faciliter la lecture, nous appelons abusivement dans la suite ``{\it œuvre}'' tout objet présenté dans un musée, tout en sachant que l'on dépasse le cadre strict des artefacts (i.e. objets fabriqués par l'Homme), comme les musées dédiés aux roches par exemple.
Dans ce travail, nous nous intéressons à un corpus d'instances d'objets, plutôt que de classes d'objets, car cela correspond à l'application que nous voulons faire, à savoir identifier des œuvres uniques.

Lorsque l'on se situe dans le contexte classique de recherche d'information, Spärk-Jones et van Rjsbergen ont défini en 1975~\cite{sparkrijs1975}, des recommandations pour une collection de test ``idéale'', parmi lesquelles :
\begin{itemize}
\item Faciliter et promouvoir la recherche : de tels corpus doivent permettre à la communauté de progresser;
\item Être de taille suffisante pour permettre des évaluations suffisamment réalistes pour répondre à de nombreux besoins en information;
\item Que la collection soit définie dans un projet spécifique avec des créateurs expérimentés : les données générées dans de tels cadres ont davantage de chance d'être adaptés à une tâche claire et bien définie;
\item Que la collection fasse l'objet de maintenance, qu'elle soit correctement documentée et accessible : ceci permet de garantir la pérennité des données (corrections d'erreurs, extension de la collection, etc.).
\end{itemize}

Nous focalisons ici notre attention sur la construction de corpus d'images pour l'indexation et la recherche {\it ad-hoc}. Dans ce cadre, les éléments que nous choisissons de caractériser sont les suivants :
\begin{itemize}
\item {\bf Le type d'objets :} l'un des éléments clés à prendre en compte est le type d'objets présents dans un musée donné. Grossièrement, nous classons ces œuvres suivant qu'elles sont 2D ou 3D: 2D pour les peintures, les photographies, 3D pour les sculptures, les objets architecturaux, etc.;
\item {\bf Le type de support :} Il est important de savoir si une collection ne comporte que des images, des vidéos, ou les deux, et ceci pour le corpus ou bien les requêtes;
\item {\bf L'acquisition de la collection :} c'est le passage de l'objet au support. La manière dont est acquis le corpus impacte grandement la qualité des recherches scientifiques l'utilisant. Si cette acquisition est bien contrôlée (e.g., qualité de prise de vue constante, luminosité stable, etc.), il est alors plus facile à la fois de conclure sur les apports de nouvelles approches que sur leurs limites. Dans les corpus traités ici, nous avons tenté de maintenir une certaine homogénéité au niveau des images et des vidéos ;
\item {\bf La taille de la collection :} en vue d'une utilisation scientifique, il faut garantir une taille suffisante du corpus (nombre d'objets, nombre d'images) et de l'ensemble de requêtes. Si le corpus est trop petit, il est alors difficile d'obtenir des statistiques fiables et significatives des différences entre les systèmes comparés à l'aide de ce corpus. Dans notre cas, nous garantissons plusieurs images exemples (entre 3 et 12) par œuvre considérée, afin de pouvoir estimer la robustesse des systèmes;
\end{itemize}
Nous gardons également à l'esprit l'intérêt d'une collection pour la communauté scientifique: il dépend de nombreux paramètres, comme la difficulté intrinsèque de la collection (objets), sa variabilité (objets, supports, prises de vues, ...), sa similarité avec des données réalistes. Si par exemple une approche de l'état de l'art classique est capable d'atteindre un taux de reconnaissance de 100\% sur une collection, elle n'a pas vraiment de valeur ajoutée pour la communauté. Ceci ne présage cependant pas d'emplois ``détournés'' d'une telle collection. Un bon exemple est donné par~\cite{Chatfield2015} : les auteurs étendent une collection relativement petite, afin de rendre plus difficile la reconnaissance de photographies de bâtiments. Des données variées dans une collection de test (aussi bien au niveau du corpus ou des requêtes) sont très utiles pour estimer la robustesse d'une approche. La notion de réalisme est un élément qui est également primordial, et ceci peut porter sur les données du corpus ou bien sur les requêtes, suivant l'objectif visé.
