\begin{savequote}[75mm] 
Nulla facilisi. In vel sem. Morbi id urna in diam dignissim feugiat. Proin molestie tortor eu velit. Aliquam erat volutpat. Nullam ultrices, diam tempus vulputate egestas, eros pede varius leo.
\qauthor{Quoteauthor Lastname} 
\end{savequote}

\chapter{Apprentissage fin et transfert de connaissance}

Apprentissage fin et transfert de connaissances reposent sur le fait d'apprendre dans un premier temps sur un corpus différent.
Cette technique permet de contourner les difficultés que soulève un corpus trop petit, où sur lequel on ne peut pas apprendre complètement un réseau de neuronnes. 
Dans notre cas, nos corpus n'ont pas la taille suffisante pour faire convergé un réseau de plusieurs millions de paramêtres. Cependant, même si nous agrandissions ces corpus avec de nouvelles données, un autre problème reste : le manque de diversité.
Les images que nous prenons viennent de musée, où la variabilité des prises de vu, de lumière et de distances sont limité. Contrairement à des corpus de reconnaissance de classe, il n'y a pas de diversité à l'intérieur de 
chaque classe (ici instance). Donc même en prenant des centaines de photos de chaque oeuvre, rien ne garantit que cela représente suffisamment de données effective pour entraîner un réseau profond.



\section{Transfert de connaissance pour des corpus réduits}

\section{Apprentissage fin}

