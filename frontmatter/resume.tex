% le résumé
\justify
%Contexte 
Cette thèse adresse le problème d'accès à l'information en mobilité.
On s'intéresse à comment rendre l'information à propos des œuvres accessible automatiquement aux visiteurs de lieux touristiques.
Elle s'inscrit dans le cadre du projet GUIMUTEIC, qui vise à équiper les visiteurs de musées d'un outil d'aide à l'accès à l'information en mobilité.
Être capable de déterminer si le visiteur désire avoir accès à l'information signifie identifier le contexte autour de lui, afin de fournir une réponse adaptée, et réagir à ses actions.

%Problématiques
Ce travail est lié aux problématiques d'identification de points d'intérêts, pour déterminer le contexte, et d'identification de gestes des utilisateurs, pour répondre à leurs demandes. 
Dans le cadre du notre projet, le visiteur est donc équipé d'une caméra embarquée.
L'objectif est de fournir une solution à l'aide à la visite, en développant des méthodes de vision pour l'identification d'objet, et de détection de gestes dans les vidéos à la première personne. 

%Contributions
Nous proposons dans cette thèse une étude de la faisabilité et de l'intérêt de l'aide à la visite, ainsi que de la pertinence des gestes dans le cadre de l'interaction avec un système embarqué. 
Nous définissons une nouvelle approche pour l'identification d'objets grâce à des réseaux de neurones profonds siamois pour l'apprentissage de similarité entre les images, avec apprentissage des régions d'intérêt dans l'image. 
Nous explorons également l'utilisation de réseaux à taille réduite pour le détection de gestes en mobilité. 
Nous présentons pour cela une architecture utilisant de nouveaux types de bloc de convolutions, pour réduire le nombre de paramètres du réseau et permettre son utilisation sur processeur mobile. 
Pour évaluer nos propositions, nous nous appuyons sur plusieurs corpus de recherche d'image et de gestes, créés spécialement pour le projet.

