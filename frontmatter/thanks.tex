% the acknowledgments section

Après avoir passé trois ans au Laboratoire d'Informatique de Grenoble, pour la réalisation de mon travail de thèse, j'aimerais remercier toutes les personnes qui m'ont aidé dans la réalisation de ce projet.
Je tiens tout d'abord à remercier mes directeurs de thèse, Philippe Mulhem et Jean-Pierre Chevallet. 
Nous avons passé trois années à travailler ensemble avec un très bonne entente, et une ambiance chaleureuse. 
Les longs débats enrichissants et les discussions plus décontractées passés ensemble ont été un plaisir.
C'est grâce à eux que j'ai pu réaliser cette thèse dans les meilleurs conditions, grâce à leur aide, leur support et leur amitié.

Je remercie en particulier les professeurs Véronique Eglin et Lynda Tamine-Lechani d'avoir consacrer du temps à lire et à rapporter mon travail de thèse.
Leur travail de lecture approfondie de ce manuscrit a permis une analyse détaillée et des commentaires enrichissants sur ce travail. 
J'aimerais également exprimer ma gratitude aux professeurs Hervé Glotin et Denis Pellerin pour avoir accepter d'examiner mon travail et de participer à ma soutenance.

Je voudrais également remercier tous les membres de l'équipe MRIM avec lesquels j'ai pu travailler, directement ou non. Georges Quénot, Catherine Bérut et Lorraine Goeriot ont participé à rendre l'équipe vivante et intéressante, d'un point de vue scientifique et sociale.
Les doctorants de l'équipe MRIM, Anuvabh Dutt, Jibril Frej, Nawal Ould-Amer et Seydoux Doubia, ont été très présents avec moi et les discussions que nous avons pu avoir ont été d'une grande aide dans la réalisation de ce travail.
Les personnes que j'ai encadré, Matthias Kohl et Marion Schmitt, ont apportés à l'équipe et à moi-même de nouvelles idées, du dynamisme, et je les en remercie.

Les laboratoire d'informatique de Grenoble à été un endroit agréable où travailler, et pour cela je remercie tous les doctorants et ingénieurs avec qui j'ai pu partager repas, pauses café, soirées, et toutes les activités sociales qui ont rendu cette expérience très agréable.
Je remercie particulièrement : Sami Alkhoury, Belen Baez, Anil Goyal, Vera Shalaeva, Karim Assaad, Alexandre Berard, Carole Plasson, Carole Adam, Tien Nguyen, Emeric Grange et Baptiste Vernier. 
Je remercie Dominique Vaufreydaz, qui a été un ami, un collègue, et une personne sur qui j'ai pu compter depuis que j'ai commencé l'informatique à Grenoble et qui m'a donné le goût de la recherche et de l'enseignement.

Les personnes qui ont rendu tout cela possible, sont, bien entendu, ma famille. Mes parents m'ont soutenu et encouragé. Ma compagne, Radhia, a été un support psychologique et scientifique indispensable de chaque instant. Je remercie toute ma famille, Maud, Lilian, Jamila et Ilyes. 
